%%%%%%%%%%%%%%%%%%%%%%%%%%%%%%%%%%%%%%%%%%%%%%%%%%%%%%%%%%%%%%%%%%%%%%
%%	Name: "Signal analysis template"
%%	File name: signalanalysis_template_main
%%	Version: 1.5
%%
%%	Compiler: XeLaTeX
%%
%%%%%%%%%%%%%%%%%%%%%%%%%%%%%%%%%%%%%%%%%%%%%%%%%%%%%%%%%%%%%%%%%%%%%%

\documentclass[conference,compsoc,onecolumn]{IEEEtran}

% *** LANGUAGE UTILITY PACKAGES ***
\usepackage[utf8]{inputenc} % Required for including letters with accents
\usepackage[spanish]{babel}
\usepackage{hyperref}
\usepackage{graphicx}
% *** USED PACKAGES ***
\include{usedpackages}

% *** CARPETA DONDE SE GUARDARAN LAS IMAGENES ***
\graphicspath{{figures/}}

% *** NUEVOS COMANDOS Y CONFIGURACIONES VARIAS ***
\interdisplaylinepenalty=2500
\newcommand{\Lpagenumber}{\ifdim\textwidth=\linewidth\else\bgroup
	\dimendef\margin=0
	\ifodd\value{page}\margin=\oddsidemargin
	\else\margin=\evensidemargin
	\fi
	\raisebox{\dimexpr -\topmargin-\headheight-\headsep-0.5\linewidth}[0pt][0pt]{%
		\rlap{\hspace{\dimexpr \margin+\textheight+\footskip}%
			\llap{\rotatebox{90}{\thepage}}}}%
	\egroup\fi}

\AddEverypageHook{\Lpagenumber}%

\newcommand{\newtxt}[1]{\textcolor{black}{#1}}
\renewcommand\IEEEkeywordsname{Palabras cláve:}
\newcommand{\mx}[1]{\mathbf{\bm{#1}}} % Matrix command
\newcommand{\vc}[1]{\mathbf{\bm{#1}}} % Vector command

%% Separación de palabras
\hyphenation{op-tical net-works semi-conduc-tor HHMMSS}


\begin{document}

% *** TITLES AND NAMES ***
% title of the document
\title{Proyecto de aula: Plataforma de seguimiento de datos COVID-19 para Colombia}
% author names and affiliations
\makeatletter
\newcommand{\linebreakand}{%
    \end{@IEEEauthorhaling}
    \hfill\mbox{}\par
    \mbox{}\hfill\begin{@IEEEauthorhalign}
}
\makeatother
%%%%
\author{\IEEEauthorblockN{Lady~Geraldine~Salazar~Bayona}
\IEEEauthorblockA{Escuela de Ciencias Exactas\\
Universidad Sergio Arboleda-Bogotá, Colombia\\
lady.salazar01@correo.usa.edu.co}
\and
\IEEEauthorblockN{Juan~Pablo~Mora~Aragón}
\IEEEauthorblockA{Escuela de Ciencias Exactas\\
Universidad Sergio Arboleda-Bogotá, Colombia\\
juan.mora03@correo.usa.edu.co}
\linebreakand
\IEEEauthorblockN{Jessica~Valentina~Parrado~Alfonso}
\IEEEauthorblockA{Escuela de Ciencias Exactas\\
Universidad Sergio Arboleda-Bogotá, Colombia\\
jessica.parrado01@correo.usa.edu.co}}

% *** MAKE TITLE ***
\maketitle
\IEEEoverridecommandlockouts
\IEEEpeerreviewmaketitle

\begin{abstract}
En el presente laboratorio se aplicarán los procesos necesarios para extraer información de páginas web con datos actualizados sobre la pandemia actual, esto se hará usando Python el cual permite de manera sencilla la extracción, interpretación y la realización de las gráficas. Es de destacar que para el proceso de adquirir la información se hace necesario una amplia investigación para poder hallar la base de datos adecuada. Los resultados y el proceso se podrán visualizar a lo largo del proyecto detalladamente.
\end{abstract}


\begin{IEEEkeywords}
    \LaTeX, PDF, Proyecto, Análisis de datos, Gráficas, Python, Pandas, WebScraping, Pandemia.
\end{IEEEkeywords}


\section{Marco teórico}
\label{sec:introduction}

\textbf{Pandemia actual Covid-19.}
\\\

\justify
\\\

COVID‑19 es la enfermedad infecciosa causada por el coronavirus que se ha descubierto más recientemente. Tanto este nuevo virus como la enfermedad que provoca eran desconocidos antes de que estallara el brote en Wuhan (China) en diciembre de 2019. Actualmente la COVID‑19 es una pandemia que afecta a muchos países de todo el mundo entre esos Colombia. Por lo pronto, se destaca la estrategia de Colombia y que ha sido referente a nivel mundial, buscando siempre aplanar la curva con la implicación de extender la epidemia un poco. "Hacerla más prolongada, de menor impacto, buscando una mayor inmunidad en la población, entonces digamos que la apuesta ha sido esa".Estas medidas se tomaron con el conocimiento de lo que paso en Europa, con Italia y España principalmente. De ahí en adelante Colombia ha venido tomando decisiones que ya comprenden tres fases y aperturas graduales, que, si se comparan con las de otros países de la región, nuestro país lo ha hecho bien, ya que no ha habido la necesidad de retroceder. .~\cite{1}
\\\

\textbf{Python}
\\\

Python es un lenguaje de programación interpretado de tipado dinámico Se trata de un lenguaje de programación creado en 1991 por Guindo Van Rossum (1956, Holanda). Python es una opción interesante para realizar todo tipo de programas que se ejecuten en cualquier máquina..~\cite{3}
\\\

Para realizar este proyecto es necesario importar los siguientes paquetes:
\\\

\begin{itemize}
\item import pandas as pd: Pandas es una librería de python destinada al análisis de datos, que proporciona unas estructuras de datos flexibles y que permiten trabajar con ellos de forma muy eficiente.~\cite{4}
\item import matplotlib.pyplot as plt: Matplotlib es probablemente el paquete de Python más utilizado para gráficos 2D. Proporciona una manera muy rápida de visualizar datos y figuras con calidad de publicación en varios formatos.~\cite{5}
\item import os: El módulo os de Python le permite a usted realizar operaciones dependientes del Sistema Operativo como crear una carpeta, listar contenidos de una carpeta, conocer acerca de un proceso, finalizar un proceso, etc.~\cite{6}
\item import requests: requests es una librería Python que facilita enormemente el trabajo con peticiones HTTP. Antes o después, en algún proyecto, es posible que tengas que hacer peticiones web, ya sea para consumir un API, extraer información de una página o enviar el contenido de un formulario de manera automatizada.~\cite{7}
\end{itemize}
\\\

\textbf{EXTRACCIÓN DE DATOS}
\\\

Para lograr un uso adecuado de la información para los fines propuestos, primero tenemos que saber la forma de extraer datos, organizarlos, para luego poder trabajar con ellos.
\\\

 \textsl{CSV}: Un csv (comma-separated values) es un archivo de texto que almacena los datos en forma de columnas, separadas por coma y las filas se distinguen por saltos de línea.~\cite{8}
 \\\

\textbf{WebScraping}
\\\

El web scraping consiste en navegar automáticamente una web y extraer de ella información. Esto puede ser muy útil para muchísimas cosas y beneficioso para casi cualquier negocio.~\cite{9}
\\\

\textbf{PANDAS}
\\\

Pandas es un paquete de Python que proporciona estructuras de datos similares a los dataframes de R. Pandas depende de Numpy, la librería que añade un potente tipo matricial a Python. Los principales tipos de datos que pueden representarse con pandas son: Datos tabulares con columnas de tipo heterogéneo con etiquetas en columnas y filas y Series temporales
\\\

\section{Resultados}
\label{sec:results}

Para la realización de este proyecto se utilizó el siguiente código:
\\\

Primero se deben importar las librerías necesarias para la ejecución correcta del programa:
\\\

\begin{lstlisting}
import csv
import os
import urllib
import requests
import pandas as pd
import datetime
import matplotlib.pyplot as plt
import numpy as np
from numpy.linalg import inv
import plotly.express as px
import plotly
from tkinter import *
from tkinter import messagebox as MessageBox

\end{lstlisting}
\\\
Luego, se utiliza el siguiente código para ingresar al sitio web en donde están los datos necesarios para el proyecto (tanto de Bogotá y Colombia) y así poder descargar de manera local los archivos CSV con todos los datos.\\\

\begin{lstlisting}
#Ingresar al sitio y descargar el CSV con los datos de Colombia
url="https://www.datos.gov.co/api/views/gt2j-8ykr/rows.csv?accessType=DOWNLOAD&bom=true&format=true"
response = requests.get(url)
with open(os.path.join("Archivo", "DataColombia.csv"), "wb") as f:
    f.write(response.content)

#Ingresar al sitio y descargar el CSV con los datos de Bogotá
url="https://datosabiertos.bogota.gov.co/dataset/44eacdb7-a535-45ed-be03-16dbbea6f6da/resource/b64b...
a3c4-9e41-41b8-b3fd-2da21d627558/download/osb_enftransm-covid26102020.csv"
response = requests.get(url)
with open(os.path.join("Archivo", "DataBogota.csv"), "wb") as f:
    f.write(response.content)

\end{lstlisting}
\\\

\textbf{CÓDIGO PARA DATOS COVID COLOMBIA}
\\\

Para el caso de Colombia  se abre el archivo CSV local y se obtienen todos los datos. Además, se hace el arreglo de los nombres de las columnas y se establece que para la columna del "Sexo" de la persona debe ser "M" o "F".
\\\
\begin{lstlisting}
#///////////////////COLOMBIA///////////////////////////////////////
#Abrir el CSV que se descargó de Colombia y obtener la información

data =pd.read_csv('Archivo/DataColombia.csv', parse_dates=[0], dayfirst=True)


data.head()

data.columns=['Fecha', 'ID', 'Fecha2', 'Código DIVIPOLA', 'Departamento', 'Código DIVIPOLA2', 'Ciudad','Edad','Unnidad', 'Sexo', 'Tipo', 'Ubicacion','Atencion', 'Código ISO del país','Nombre del país','Recuperado','Fecha de inicio de síntomas','Fecha de muerte','Fecha de diagnóstico','Fecha de recuperación','Tipo de recuperación','Pertenencia étnica','Nombre del grupo étnico']

d={'m':'M', 'f':'F','F':'F', 'M':'M'}
data['Sexo']=data['Sexo'].apply(lambda x:d[x])

data['Fecha'] = pd.to_datetime(data['Fecha'])
\end{lstlisting}
\\\

Después se realiza el código para crear las 4 gráficas de Barras y las 2 gráficas de Tortas, en donde se busca representar los datos sobre los Departamentos, Sexo, Recuperados, Fallecido y según su condición.
\\\

\begin{lstlisting}
#GRÁFICAS DE BARRAS

fig1=plt.figure(figsize=(12,6))
data.Departamento.value_counts().plot(kind='bar', alpha=0.5)
plt.title('Número de Casos Totales según Departamento en Colombia')
plt.show()

fig=plt.figure(figsize=(12,6))
data.Sexo.value_counts().plot(kind='bar', alpha=0.5)
plt.title('Número de Casos totales en Colombia según el Sexo')
plt.show()

fig2=plt.figure(figsize=(12,6))
data.Recuperado[data.Recuperado == "Recuperado"].value_counts().plot(kind='bar', alpha=0.5)
plt.title('Número de Casos de Recuperados en Colombia')
plt.show()

fig3=plt.figure(figsize=(12,6))
data.Recuperado[data.Recuperado == "Fallecido"].value_counts().plot(kind='bar', alpha=0.5)
plt.title('Número de Casos de Fallecidos en Colombia')
plt.show()

#GRÁFICAS DE TORTAS

fig1=plt.figure(figsize=(12,6))
plt.pie(data.Sexo.value_counts(), autopct="%1.1f%%", shadow=True, radius= .9)
plt.title('Porcentaje de Casos totales en Colombia según el Sexo', bbox={"facecolor":"0.8","pad":5})
plt.legend( labels= data.Sexo.value_counts().index.unique(), loc='upper right')
plt.show()

fig1=plt.figure(figsize=(12,6))
plt.pie(data.Recuperado[data.Recuperado != "fallecido"].value_counts(), autopct="%1.1f%%", shadow=True, radius= .999)
plt.title('Porcentaje de la situación de los contagiados en Colombia', bbox={"facecolor":"0.8","pad":5})
plt.legend( labels=data.Recuperado.value_counts().index.unique(), loc='upper right')
plt.show()

\end{lstlisting}
\\\

Al ejecutar todo el código se obtienen las siguientes gráficas:
\\\

\begin{figure}[htbp]
\centering
\subfigure[\scriptsize Gráfica Número de Casos Totales según el Departamento en Colombia]{\includegraphics[width=50mm]{Figures/Figure_1.png}}
\subfigure[\scriptsize Gráfica Número de Casos según el Sexo en Colombia]{\includegraphics[width=50mm]{Figures/Figure_2.png}}
\subfigure[\scriptsize Gráfica Número de Casos Recuperados en Colombia]{\includegraphics[width=50mm]{Figures/Figure_3.png}}
\label{fig:lego}
\end{figure}

\begin{figure}[htbp]
\centering
\subfigure[\scriptsize Gráfica Número de Casos Fallecidos en Colombia]{\includegraphics[width=50mm]{Figures/Figure_4.png}}
\subfigure[\scriptsize Gráfica Porcentajes de Casos Según el Sexo ]{\includegraphics[width=60mm]{Figures/Figure_5.png}}
\subfigure[\scriptsize Gráfica Porcentajes de Casos Según su situación]{\includegraphics[width=60mm]{Figures/Figure_6.png}}
\caption{ Gráficas Resultantes} \label{fig:lego}\label{fig:lego}
\end{figure}
\\\

Finalmente, se añade el código para ejecutar el mapa de calor de Colombia. Para esto se utilizó un enlace externo donde están los datos GeoJson con los polígonos de cada Departamento.De hecho, se utilizó la librería  "plotly.express" la cual nos permitió crear el mapa teniendo en cuenta el archivo GeoJson y el archivo CSV.
\\\

\begin{lstlisting}
#//////////////////MAPA DE CALOR COLOMBIA//////////////////////////
repo_url = 'https://gist.githubusercontent.com/john-guerra/43c7656821069d00dcbc/raw/be6a6e239cd5b5b803c6e...
7c2ec405b793a9064dd/Colombia.geo.json' #Archivo GeoJSON
mx_regions_geo = requests.get(repo_url).json()
extra=np.array(data.Departamento.value_counts())
casosDepartamentos=np.array([extra[1],extra[6],extra[0],extra[24],extra[17],extra[19],extra[21]...
,extra[18],extra[9],extra[7],extra[3],extra[28],extra[12],extra[22],extra[25],extra[10],extra[11]...
,extra[13],extra[23],extra[15],extra[4],extra[16],extra[14],extra[2],extra[29],extra[26],extra[27]...
,extra[30],extra[33],extra[32],extra[34],extra[35],extra[31]])
nombresDepartamentos=np.array(["ANTIOQUIA", "ATLANTICO", "SANTAFE DE BOGOTA D.C", "BOLIVAR", "BOYACA", "CALDAS", "CAQUETA", "CAUCA", "CESAR", "CORDOBA", "CUNDINAMARCA", "CHOCO", "HUILA", "LA GUAJIRA", "MAGDALENA", "META", "NARIÑO", "NORTE DE SANTANDER", "QUINDIO", "RISARALDA", "SANTANDER", "SUCRE", "TOLIMA", "VALLE DEL CAUCA", "ARAUCA", "CASANARE", "PUTUMAYO", "AMAZONAS", "GUAINIA", "GUAVIARE", "VAUPES", "VICHADA", "ARCHIPIELAGO DE SAN ANDRES PROVIDENCIA Y SANTA CATALINA"])

fig = px.choropleth(data_frame=data,
                    geojson=mx_regions_geo,
                    locations=nombresDepartamentos, # nombre de la columna del Dataframe
                    featureidkey='properties.NOMBRE_DPT',  # ruta al campo del archivo GeoJSON con el que se hará la relación (nombre de los estados)

                    color=casosDepartamentos, #El color depende de las cantidades
                    color_continuous_scale="burg", #greens
                    #scope="north america"
                   )

fig.update_geos(showcountries=False, showcoastlines=False, showland=False, fitbounds="locations")

fig.update_layout(
    title_text = 'Casos de infección en Colombia',
    font=dict(
        #family="Courier New, monospace",
        family="Ubuntu",
        size=18,
        color="#7f7f7f"
    )
)
plotly.offline.plot(fig)

\end{lstlisting}
\\\

Entregándonos la siguiente gráfica:
\\\

\begin{figure}[htbp]
\centering
\subfigure[\scriptsize ]{\includegraphics[width=60mm]{Figures/Figure_7.png}}
\caption{ Mapa de Calor Colombia} \label{fig:lego}
\end{figure}
\\\

\textbf{CÓDIGO PARA DATOS COVID BOGOTÁ}
\\\

Para el caso de Bogotá  se abre el archivo CSV local y se obtienen todos los datos. Además, se hace el arreglo de los nombres de las columnas, se elimina la primera fila que no entrega información relevante y se establece los rangos de edad que se tendrán en cuenta en las gráficas.
\\\

\begin{lstlisting}
#///////////////////BOGOTÁ//////////////////////////////////////
#Abrir el CSV que se descargó de Bogotá y obtener la información

dato =pd.read_csv('Archivo/DataBogota.csv', delimiter=";", encoding='iso-8859-1',names=['Fecha_Sintomas', 'FechaDiagnostico', 'Ciudad', 'Localidad', 'Edad', 'Uni_Med', 'Sexo', 'Fuente_Contagio', 'Ubicacion', 'Estado'])
dato.head()
dato=dato.drop([0], axis=0) #Borrar la primera fila que tiene texto innecesario
dato.Edad=dato.Edad.astype(float) #Convertir la columna Edad a float
age_groups = pd.cut(dato.Edad, bins=[19, 40, 65, np.inf]) #Rangos de edad
\end{lstlisting}
\\\

Después se realiza el código para crear las 6 gráficas de Barras, las 2 gráficas de Tortas y las 3 gráficas de área, en donde se busca representar los datos sobre los Localidades, Sexo, Recuperados, Fallecidos, Edad y según su condición.
\\\

\begin{lstlisting}

#GRÁFICAS DE BARRAS

fig=plt.figure(figsize=(12,6))
dato.Sexo[dato.Sexo != "SEXO"].value_counts().plot(kind='bar', alpha=0.5)
plt.title('Número de Casos totales en Bogotá según el Sexo')
plt.show()

fig2=plt.figure(figsize=(12,6))
dato.Estado[dato.Estado == "Recuperado"].value_counts().plot(kind='bar', alpha=0.5)
plt.title('Número de Casos de Recuperados en Bogotá')
plt.show()

fig3=plt.figure(figsize=(12,6))
dato.Estado[dato.Estado == "Fallecido"].value_counts().plot(kind='bar', alpha=0.5)
plt.title('Número de Casos de Fallecidos en Bogotá')
plt.show()

fig1=plt.figure(figsize=(12,6))
dato.Localidad[dato.Localidad != "Sin dato"].value_counts().plot(kind='bar', alpha=0.5)
plt.title('Número de Casos Totales según Localidad en Bogotá')
plt.show()

fig1=plt.figure(figsize=(12,6))
pd.crosstab(dato.Localidad[dato.Localidad != "Sin dato"], dato['Sexo']).plot(kind='bar', alpha=0.5)
plt.title('Número de Casos totales en Bogotá según el Sexo y la Localidad')
plt.show()

fig1=plt.figure(figsize=(12,6))
pd.crosstab(dato.Localidad[dato.Localidad != "Sin dato"], age_groups).plot(kind='bar', alpha=0.5)
plt.title('Número de Casos totales en Bogotá según el rango de edad y la Localidad')
plt.show()

#GRÁFICAS DE TORTAS
fig1=plt.figure(figsize=(12,6))
plt.pie(dato.Sexo[dato.Sexo != "SEXO"].value_counts(), autopct="%1.1f%%", shadow=True, radius= .9)
plt.title('Porcentaje de Casos totales en Bogotá según el Sexo', bbox={"facecolor":"0.8","pad":5})
plt.legend( labels= dato.Sexo.value_counts().index.unique(), loc='upper right')
plt.show()

fig1=plt.figure(figsize=(12,6))
plt.pie(dato.Estado[dato.Estado != "ESTADO"].value_counts(), autopct="%1.1f%%", shadow=True, radius= .999)
plt.title('Porcentaje de la situación de los contagiados en Bogotá', bbox={"facecolor":"0.8","pad":5})
plt.legend( labels=['%s, %1.1f%%' % (
        l, (float(s) / len(dato)) * 100) for l, s in zip(dato.Estado.value_counts().index.unique(), dato.Estado.value_counts())], loc='upper right')
plt.show()

#GRÁFICAS EXTRAS BOGOTÁ

fig1=plt.figure(figsize=(12,6))
dato.Localidad[dato.Localidad != "Sin dato"].value_counts().plot(kind='area', alpha=0.5)
plt.title('Número de Casos Totales según Localidad en Bogotá')
plt.show()

fig1=plt.figure(figsize=(12,6))
pd.crosstab(dato.Localidad[dato.Localidad != "Sin dato"], dato['Sexo']).plot(kind='area', alpha=0.5)
plt.title('Número de Casos totales en Bogotá según el Sexo y la Localidad')
plt.show()

fig1=plt.figure(figsize=(12,6))
pd.crosstab(dato.Localidad[dato.Localidad != "Sin dato"], age_groups).plot(kind='area', alpha=0.5)
plt.title('Número de Casos totales en Bogotá según el rango de edad y la Localidad')
plt.show()
\end{lstlisting}
\\\

Al ejecutar todo el código se obtienen las siguientes gráficas:
\\\

\begin{figure}[htbp]
\centering
\subfigure[\scriptsize Gráfica Número de Casos Totales según el Sexo en Bogotá]{\includegraphics[width=50mm]{Figures/1Figure_1.png}}
\subfigure[\scriptsize Gráfica Número de Casos Recuperados en Bogotá Colombia]{\includegraphics[width=50mm]{Figures/1Figure_2.png}}
\subfigure[\scriptsize Gráfica Número de Fallecidos en Bogotá]{\includegraphics[width=50mm]{Figures/1Figure_3.png}}
\label{fig:lego}
\end{figure}

\begin{figure}[htbp]
\centering
\subfigure[\scriptsize Gráfica Número de Casos Totales según la Localidad en Bogotá]{\includegraphics[width=60mm]{Figures/1Figure_4.png}} \label{fig:lego}
\subfigure[\scriptsize Gráfica Número de Casos Totales en Bogotá según el Sexo y la Localidad]{\includegraphics[width=40mm]{Figures/1Figure_5.png}}
\subfigure[\scriptsize Gráfica Número de Casos Totales en Bogotá según la Edad y la Localidad]{\includegraphics[width=40mm]{Figures/1Figure_6.png}}
\caption{ Gráficas en estilo de Barras Verticales}
\end{figure}
\\\

\begin{figure}[htbp]
\centering
\subfigure[\scriptsize Gráfica Porcentajes de Casos Según el Sexo en Bogotá ]{\includegraphics[width=50mm]{Figures/1Figure_7.png}}
\subfigure[\scriptsize Gráfica Porcentajes de Casos Según su situación en Bogotá]{\includegraphics[width=50mm]{Figures/1Figure_8.png}}
\caption{ Gráficas en estilo de Torta} \label{fig:lego}
\end{figure}
\\\

\begin{figure}[htbp]
\centering
\subfigure[\scriptsize Número de Casos Totales Según la Localidad en Bogotá ]{\includegraphics[width=50mm]{Figures/1Figure_9.png}}
\subfigure[\scriptsize Número de Casos Totales Según la Localidad y el Sexo en Bogotá]{\includegraphics[width=40mm]{Figures/1Figure_10.png}}
\subfigure[\scriptsize Número de Casos Totales Según la Localidad y la Edad en Bogotá]{\includegraphics[width=40mm]{Figures/1Figure_11.png}}
\caption{ Gráficas en estilo Área} \label{fig:lego}
\end{figure}
\\\

Finalmente, se añade el código para ejecutar el mapa de calor de Bogotá. Para esto se utilizó un enlace externo donde están los datos GeoJson con los polígonos de cada localidad.De hecho, se utilizó la librería  "plotly.express" la cual nos permitió crear el mapa teniendo en cuenta el archivo GeoJson y el archivo CSV.
\\\

\begin{lstlisting}

#//////////////////MAPA DE CALOR BOGOTÁ//////////////////////////
repo_url2 = 'https://raw.githubusercontent.com/JessicaParrado/Localidades/main/bogota_localidades.geojson' #Archivo GeoJSON
bo_regions_geo = requests.get(repo_url2).json()
extra2=np.array(dato['Localidad'].value_counts())
casosLocalidades=np.array([extra2[4], extra2[1], extra2[7], extra2[0], extra2[11], extra2[18], extra2[16], extra2[14], extra2[9], extra2[2], extra2[20], extra2[13], extra2[5], extra2[15], 1345, extra2[8], extra2[6], extra2[12], extra2[3], extra2[10]])
nombresLocalidades=np.array(["CIUDAD BOLIVAR","SUBA", "RAFAEL URIBE URIBE", "KENNEDY", "USME", "LOS MARTIRES", "SANTA FE", "BARRIOS UNIDOS", "FONTIBON", "ENGATIVA", "CANDELARIA", "CHAPINERO", "ANTONIO", "TEUSAQUILLO", "SUMAPAZ", "SAN CRISTOBAL", "USAQUEN", "TUNJUELITO", "BOSA", "PUENTE ARANDA"])

fig = px.choropleth(data_frame=dato,
                    geojson=bo_regions_geo,
                    locations=nombresLocalidades, # nombre de la columna del Dataframe
                    featureidkey='properties.NOMBRE',  # ruta al campo del archivo GeoJSON con el que se hará la relación (nombre de los estados)

                    color=casosLocalidades, #El color depende de las cantidades
                    color_continuous_scale="Teal", #greens
                    #scope="north america"
                   )

fig.update_geos(showcountries=True, showcoastlines=True, showland=True, fitbounds="locations")

fig.update_layout(
    title_text = 'Casos de infección en Bogotá',
    font=dict(
        #family="Courier New, monospace",
        family="Ubuntu",
        size=18,
        color="#7f7f7f"
    )
)
plotly.offline.plot(fig)
\end{lstlisting}
\\\

Entregándonos la siguiente gráfica:
\\\

\begin{figure}[htbp]
\centering
\subfigure[\scriptsize Mapa de Calor de Bogotá]{\includegraphics[width=60mm]{Figures/1Figure_12.png}}
\label{fig:lego}
\end{figure}
\\\

\textbf{CÓDIGO PROYECCIÓN CASOS COLOMBIA}
\\\

Para este caso se utilizó las fórmulas:
\\\

\begin{figure}[htbp]
\centering
\subfigure[\scriptsize]{\includegraphics[width=30mm]{Figures/Imagen1.png}}
\subfigure[\scriptsize]{\includegraphics[width=30mm]{Figures/Imagen7.png}}
\label{fig:lego}
\end{figure}
\\\

La cual busca encontrar el valor futuro aproximado teniendo en cuenta unos valores previos y un promedio. Todo esto con el fin de crear la recta de la regresión lineal que minimiza la distancia entre los puntos de los valores que se tienen y con la cual se puede hacer proyecciones. El código utilizado es:
\\\

\begin{lstlisting}
#///////////////////////////////////////REGRESIÓN DE DATOS//////////////////////////////////////////////////////

#Graficar la información que se tiene como puntos
plt.scatter(data.Fecha[data.Fecha.dt.month != 12.0].dt.month.value_counts().index.unique(),data.Fecha[data.Fecha.dt.month != 12.0].dt.month.value_counts());
plt.xlabel('Meses');
plt.ylabel('Número de Casos')
plt.show()
#Crear el modelo
X=np.array([np.ones(len(data.Fecha[data.Fecha.dt.month != 12.0].dt.month.value_counts().index.unique())),data.Fecha[data.Fecha.dt.month != 12.0].dt.month.value_counts().index.unique()]).T
a= inv(X.T @ X ) @X.T @ data.Fecha[data.Fecha.dt.month != 12.0].dt.month.value_counts() #Fórmula Proyección

#Hacer la prediccción
x_predict=np.linspace(3,11,num=100) #Generará números del 3 al 11
subs_predict=a[0]+a[1]*x_predict #Fórmula de la recta
#Graficamos los puntos y la recta en la misma figura,
plt.scatter(data.Fecha[data.Fecha.dt.month != 12.0].dt.month.value_counts().index.unique(),data.Fecha[data.Fecha.dt.month != 12.0].dt.month.value_counts());
plt.xlabel('Meses');
plt.ylabel('Número de Casos')
plt.plot(x_predict, subs_predict,'c') #Minimiza la distancia entre los puntos
plt.show()


#Para saber número de casos al finalizar Diciembre(Mes 1 de proyección)
y= a[1]*(10)+a[0] #Fórmula de la recta, donde a[1] es la pendiente, se varia el valor de x para saber la proyección
y1=f"{y:.1f}"
#print('Al final de Diciembre del 2020, se tendrán aproximadamente '+str(y1)+' casos en Colombia')
MessageBox.showinfo("Proyección Mes 1", 'Teniendo en cuenta los resultados de los meses anteriores, se proyecta que al final de Diciembre del 2020, se tendrán aproximadamente '+str(y1)+' casos en Colombia')

#Para saber número de casos al finalizar Enero(Mes 2 de proyección)
y= a[1]*(11)+a[0] #Fórmula de la recta, donde a[1] es la pendiente, se varia el valor de x para saber la proyección
y1=f"{y:.1f}"
#print('Al final de Enero del 2021, se tendrán aproximadamente '+str(y1)+' casos en Colombia')
MessageBox.showinfo("Proyección Mes 2", 'Teniendo en cuenta los resultados de los meses anteriores, se proyecta que al final de Enero del 2021, se tendrán aproximadamente '+str(y1)+' casos en Colombia')

#Para saber número de casos al finalizar Febrero(Mes 3 de proyección)
y= a[1]*(12)+a[0] #Fórmula de la recta, donde a[1] es la pendiente, se varia el valor de x para saber la proyección
y1=f"{y:.1f}"
#print('Al final de Febrero del 2021, se tendrán aproximadamente '+str(y1)+' casos en Colombia')
MessageBox.showinfo("Proyección Mes 3", 'Teniendo en cuenta los resultados de los meses anteriores, se proyecta que al final de Febrero del 2021, se tendrán aproximadamente '+str(y1)+' casos en Colombia')
\end{lstlisting}
\\\

El cual entrega las siguientes gráficas junto con los mensajes emergentes que presentan la información de las proyecciones realizadas para Diciembre 2020, Enero 2021 y Febrero 2021:
\\\

\begin{figure}[htbp]
\centering
\subfigure[\scriptsize]{\includegraphics[width=30mm]{Figures/Imagen2.png}}
\subfigure[\scriptsize]{\includegraphics[width=30mm]{Figures/Imagen3.png}}
\label{fig:lego}
\end{figure}

\begin{figure}[htbp]
\centering
\subfigure[\scriptsize]{\includegraphics[width=50mm]{Figures/Imagen4.png}}
\subfigure[\scriptsize]{\includegraphics[width=50mm]{Figures/Imagen5.png}}
\subfigure[\scriptsize]{\includegraphics[width=50mm]{Figures/Imagen6.png}}
\label{fig:lego}
\end{figure}
\\\

\section{Conclusiones}
\label{sec:conclusions}
% Escriba su texto aquí
Este año (2020) ha llegado con muchos problemas para la sociedad de todo el mundo, la pandemia ha afectado en todos los ámbitos a las personas, por ese motivo es tan importante esta información, gracias a esto, se han podido tener planes para poder solventar esta problemática de la mejor manera, acá es cuando notamos la gran importancia de la obtención, filtración, y representación de la información. Para poder lograr esto, encontramos la información necesaria que nos dio las herramientas para poder extraer los datos, a su vez este proyecto nos ayudó a entender como poder normalizar la información para hacer un uso eficaz y poder presentar información según las necesidades actuales, todo esto con el objetivo de ser eficientes a la hora de actuar frente a esta enfermedad.
\\\

\section{Repositorio Git}
\label{sec:Repositorio}

El link del repositorio git donde está el código y los documentos es el siguiente:
\\\

https://github.com/JessicaParrado/ProyectoSenialesCorte3
\\\

\\\
\nocite{*}
\bibliographystyle{IEEEtran}
\label{sec:biblio}
% Descomente y modiffique el archivo biblio.bib para agregar bibliografía
\bibliography{bib/biblio} 





%\pagestyle{empty}
\end{document}


